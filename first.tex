\documentclass{article}

\usepackage[utf8]{inputenc}
\usepackage[x11names]{xcolor}

\title{LaTex Tutorial}
\author{Kannan Jayachandran}
\date{October 20}

\begin{document}

\maketitle

\section{Introduction}
This document contains the basic tutorial of Latex.

\section*{Formula}
We write formulas between two \$ symbols,

$$E = m \times c^2$$.

$$e^{i\pi} + 1 = 0$$
These type of equations are known as inline formulas. We can use multiple \$ to to scope the formulas effectively, also it would center the formula.

\begin{enumerate}
\item For example:
$$e =  \lim_{n \to \infty} \left(1 + \frac{1}{n}\right) ^ n = \lim_{n \to \infty} \frac{n}{\sqrt[n]{n!}}$$

\item Let's try sum:

$$e = \sum_{n=0}^ {\infty} \frac{1}{n!} $$

We can use continued fractions also:
$$ e=2+\frac{1}{1+\frac{1}{2+\frac{2}{3+\frac{3}{4+\frac{4}{5+\ddots}}}}}$$
\end{enumerate}

If we don't want numbers in the item list, we can use 
\begin{itemize}
    \item This is the first item
    \item This is the second item
\end{itemize}
 
\section*{More Stuff}

\end{document}